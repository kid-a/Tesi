\clearpage{\pagestyle{empty}\cleardoublepage}
\chapter{Architettura del sistema}
%
Il sistema di monitoraggio presentato in questa tesi \`e stato progettato 
quale possibile soluzione per una azienda installatrice di impianti 
fotovoltaici su vasta scala.
%

%
Perch\'e una azienda installatrice dovrebbe interessarsi ad un sistema 
di monitoraggio? Le motivazioni possibili sono almeno due:
%
\begin{itemize}
\item l'azienda potrebbe fornire, ai clienti che lo desiderano, un pacchetto 
      comprensivo di impianto fotovoltaico e sistema di monitoraggio; le 
      informazioni prodotte da quest'ultimo potrebbero essere utilizzate 
      per realizzare un sistema informativo in grado di dare visione 
      al cliente \Item{i} dello stato del suo impianto, \Item{ii} del 
      suo rendimento e, quindi, \Item{iii} del ritorno economico 
      rispetto all'investimento effettuato
%
\item una azienda installatrice fornisce una garanzia sull'impianto e, spesso, 
      si occupa anche della manuntenzione di quest'ultimo; un sistema di 
      monitoraggio permetterebbe di tenere sotto costante osservazione 
      lo stato degli impianti installati, a livello di componente; ci\`o
      permetterebbe una previsione e una identificazione da remoto, di 
      eventuali \emph{fault}, andando ad abbattere i costi di manutenzione
      degli impianti
\end{itemize}
%

%


%% desiderata: modularita`, elevata scalabilita`, semplicita` di installazione

%% degli approcci visti in letteratura, quale utilizzare?

\section{Impianti di Elevata Potenza}

\subsection{Definizione delle grandezze}

\begin{enumerate}
%%
\item Dati Ambientali:
\begin{itemize}
\item $T, (\deg C)$, Temperatura ambiente.
\item $R, (W/m^2)$, Irraggiamento.
\end{itemize}
%%
%%
\item Dimensionamento dell'impianto:
\begin{itemize}
\item $NI$, numero di inverter.
\item $NS_k, k \in \{1, \dots, NI\}$, numero di stringhe dell'inverter $k$.
\item $\eta _{k}, k \in \{1, \dots, NI\}$, rendimento in potenza
  dell'inverter $k$ (rapporto tra
  potenza in uscita e potenza in ingresso).
\item $NP_{j,k}, k \in \{1, \dots, NI\}, j \in \{1, \dots, NS_k\}$, numero
  di pannelli della stringa $j$ legata all'inverter $k$.
\item $\mu _{i,j,k}, k \in \{1, \dots, NI\}, j \in \{1, \dots, NS_k\}, i \in
  \{1, \dots, NP_{j,k}\}$, rendimento in potenza del pannello $i$ della stringa
  $j$ dell'inverter $k$ (rapporto tra  potenza in uscita e irraggiamento).
\item $S _{i,j,k}, k \in \{1, \dots, NI\}, j \in \{1, \dots, NS_k\}, i \in
  \{1, \dots, NP_{j,k}\}, (m^2)$, Superficie del pannello $i$ della stringa
  $j$ dell'inverter $k$.
\end{itemize}
%%
%%
\item Valori misurati a valle del quadro di AC:
\begin{itemize}
%%
\item$E_O, (KWh)$, Energia attiva totale prodotta.
\item$P_O, (KW)$, Potenza attiva.
\item$PR_O, (KW)$, Potenza reattiva.
\item$V_O, (V)$, Tensione.
\item$I_O, (A)$, Corrente.
\item$PF_O, (adimensionale)$, Sfsamento (power factor o $cos \phi$).
%%
\end{itemize}
%%
%%
\item Valori misurati a valle dell'inverter $k$:
\begin{itemize}
%%
\item$Iout_{k}, (A)$, Corrente generata.
%%
\end{itemize}
%%
%%
\item Valori \textbf{calcolati} per l'inverter $k$:
\begin{itemize}
%%
\item$Vout_{k} = V_O, (V)$, Tensione generata.
\item$Pout_{k} = Vout_{k} \cdot Iout_{k}  \cdot PF_O, (KW)$ Potenza
  (attiva) generata
\item$Pin_{k} = \frac{Pout_k}{\eta _k}, (KW)$, Potenza in ingresso
  all'inverter generata dalle stringhe.
%%
\item$Vin_{k} = \frac{Pin_k}{\sum_{j=1}^{NS_k}{Is_{j, k}}}, (V)$, Tensione
  in ingresso all'inverter
  calcolata sulla base della corrente prodotta dalle stringhe.
%%
\end{itemize}
%%
%%
\item Valori misurati a valle della stringa $j, k$:
\begin{itemize}
%%
\item$Is_{j, k}, (A)$, Corrente generata dalla stringa.
%%
\end{itemize}
%%
%%
\item Valori \textbf{calcolati} per la stringa $j, k$:
\begin{itemize}
%%
\item$Vs_{j, k} = Vin_{k}, \forall k, (V)$, Tensione generata da ogni stringa.
\item$Ps_{j, k} = Vs_{j,k} \cdot Is_{j,k}, (KW)$, Potenza generata da ogni
  stringa.
%%
\end{itemize}
\end{enumerate}
%%

%%

%% misure rilevate 
%% misure stimate 

%% dopo: come misurare?
%% dispositivi di misura utilizzati
%% sensor network
%% comunicazione over gprs o sms
