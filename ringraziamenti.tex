\chapter*{Ringraziamenti}
%
Se oggi ho la possibilit\`a di scrivere queste righe, il merito non \`e solo mio, 
ma anche, e sopratutto, di tutte quelle persone che, negli corso degli anni, mi hanno 
ispirato, consigliato, supportato, coadiuvato. In altre parole, di coloro che mi hanno fatto 
\emph{crescere}.
%

%
\emph{In primis}, vorrei ringraziare il professore Corrado Santoro per essere stato il 
mio \emph{mentore} in questi anni. Grazie per avermi ispirato e indirizzato
sempre sulla giusta strada. Grazie per avere scommesso su di me nei momenti in cui 
neanche io l'avrei fatto.
%

%
Grazie ai miei \emph{classmates} Daniele Ferro e Daniele Marletta per aver condiviso 
con me le (molte) gioie e i (pochi, fortunatamente) dolori di questi ultimi anni da 
universitari, dalle avventure in Inghilterra alle \emph{nottate} Svizzere di Swiss Eurobot.
Siete stati i \emph{migliori project collaborator degli ultimi 150 anni}.
%

%
Grazie a tutti coloro che, negli anni, sono stati miei compagni di squadra nel DIIT Team 
e nell'UNICT Team: vorrei ringraziare, in particolare, gli \emph{eurobottari} seriali, 
coloro che, come me, hanno deciso di \emph{trascurare studio, lavoro, mogli e/o fidanzate e figli} 
per dedicare la propria vita ai robot 
e a \emph{Eurobot}: Carlo Battiato, Sebastiano Gennarini, Daniele Marletta, Riccardo Massari, 
Andrea Milazzo, Rocco Milluzzo, Salvatore Pecorino, Dario Pellicori, Federico Pepe, 
Vincenzo Nicosia, Corrado Santoro, Davide Marano, Salvatore Monteleone, Alessandra Vitanza, 
Santi Passarello, Paolo Nicotra.
%
Con voi ho scoperto il piacere del lavoro \emph{in team} e del sentirsi squadra.
Grazie a voi ho scoperto sportivit\`a e senso della \emph{sfida}.
Siete stati voi, in fondo, la mia vera universit\`a.
(Adesso per\`o, basta giocare! A studiare!)
%

%
Grazie alla mia numerosa e meravigliosa famiglia, unita e forte sia nei giorni di festa, sia in 
quelli pi\`u difficili. Grazie a mio padre e mia madre per l'esempio di coraggio, dedizione e 
amore che, giornalmente, da ventiquattro anni, mi danno.
%

%
Vorrei ringraziare, infine, la signorina Carla Dipasquale, di cui sono, da tempo, un 
inguaribile ammiratore.
%
\emph{Merci pour cette belle aventure, il est temps, pour toi et moi, d'en vivre une nouvelle.}
