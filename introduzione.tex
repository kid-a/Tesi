\clearpage{\pagestyle{empty}\cleardoublepage}
\chapter*{Introduzione} 
%%\markboth{Introduzione}{Introduzione}
\addcontentsline{toc}{chapter}{Introduzione}
%

%
Contestualmente al grande aumento del numero di impianti 
fotovoltaici installati sul territorio nazionale\cite{gse2010}, 
il problema del \emph{monitoraggio} degli impianti stessi ha 
suscitato un sempre maggiore interesse.
%

%
\`E emersa, infatti, l'esigenza di implementare dei sistemi 
di raccolta e integrazione dati in grado di 
i. fornire ai \emph{soggetti responsabili} informazioni 
circa la \emph{produzione} e la \emph{rendita} dei loro impianti, 
ii. fornire agli installatori/manutentori informazioni utili 
riguardo lo stato degli impianti e riguardo le condizioni e la 
rendita dei singoli dispositivi di campo, al fine di permettere 
diagnosi \emph{remote} di eventuali anomalie o malfunzionamenti.
%

%
%% Esistono gia` dei sistemi ma sono soltanto gadget, ma vi e` necessita` di dati
%% veri, affidabili, spesso complessi, cita l'articolo di photon.

%
La tesi \`e organizzata come segue: il primo capitolo \emph{blah}.
Nel secondo capitolo \emph{blah}.
%

%
Il quinto capitolo, infine, \emph{blah}.
