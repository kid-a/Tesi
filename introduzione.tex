\clearpage{\pagestyle{empty}\cleardoublepage}
\chapter*{Introduzione} 
%%\markboth{Introduzione}{Introduzione}
\addcontentsline{toc}{chapter}{Introduzione}
%
Negli ultimi anni, contestualmente al grande aumento del 
numero di impianti fotovoltaici installati sul territorio 
nazionale\cite{gse2010}, il problema del \emph{monitoraggio} 
degli impianti stessi ha suscitato un sempre maggiore 
interesse.
%

%
\`E emersa, infatti, l'esigenza di implementare dei sistemi 
di raccolta e integrazione dati in grado di \Item{i} fornire ai 
\emph{soggetti responsabili} informazioni circa la 
\emph{produzione} e la \emph{resa} dei loro impianti, 
\Item{ii} fornire agli installatori/manutentori informazioni utili 
riguardo lo stato degli impianti e le condizioni e la 
rendita dei singoli dispositivi di campo, al fine di permettere 
diagnosi \emph{remote} di eventuali anomalie o malfunzionamenti.
%

%
Allo stato attuale, sul mercato sono gi\`a presenti numerose 
soluzioni. Tuttavia, come evidenziato in \cite{photon2010}, una 
buona parte di essi evidenzia forti caratteri di \emph{immaturit\`a}, 
specialmente per quanto riguarda la robustezza rispetto alle 
molteplici condizioni anomale che possono verificarsi.
%
Detto in altre parole, pi\`u che di sistemi di monitoraggio, 
si tratta di ottimi (e costosi) \emph{gadget}, sui cui dati 
non \`e sempre possibile fare affidamento. 
%

%
Il presente lavoro di tesi si propone di \emph{blah blah blah}

%
La tesi \`e organizzata come segue: il primo capitolo \emph{blah}.
Nel secondo capitolo \emph{blah}.
%

%
Il quinto capitolo, infine, \emph{blah}.
