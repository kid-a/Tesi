\clearpage{\pagestyle{empty}\cleardoublepage}
\chapter*{Introduzione} 
\markboth{Introduzione}{Introduzione}
\addcontentsline{toc}{chapter}{Introduzione}
%
Negli ultimi anni, contestualmente al grande aumento del 
numero di impianti fotovoltaici installati sul territorio 
nazionale\cite{gse2010}, il problema del \emph{monitoraggio} 
degli impianti stessi ha suscitato un sempre maggiore 
interesse.
%

%
\`E emersa, infatti, l'esigenza di implementare dei sistemi 
di raccolta e integrazione dati in grado di \Item{i} fornire ai 
\emph{soggetti responsabili}, informazioni circa la 
\emph{produzione} e la \emph{resa} dei loro impianti, 
\Item{ii} fornire agli \emph{installatori/manutentori} informazioni 
utili riguardo lo stato degli impianti e le condizioni e la 
rendita dei singoli dispositivi di campo, al fine di permettere 
diagnosi di eventuali anomalie o malfunzionamenti.
%

%
Allo stato attuale, sul mercato sono gi\`a presenti numerose 
soluzioni. Tuttavia, come evidenziato in \cite{photon2010}, una 
buona parte di essi evidenzia forti caratteri di \emph{immaturit\`a}, 
specialmente per quanto riguarda la robustezza rispetto alle 
molteplici condizioni anomale che possono verificarsi.
%
Detto in altre parole, pi\`u che di sistemi di monitoraggio, 
si tratta di ottimi (e costosi) \emph{gadget}, sui cui dati 
non \`e sempre possibile fare affidamento. 
%

%
Oggetto del presente lavoro di tesi sono stati il progetto e l'implementazione 
di un sistema di monitoraggio \emph{affidabile}, rapidamente ingegnerizzabile 
e commercializzabile.
%

%
La tesi \`e organizzata come segue: il primo capitolo contiene una breve 
introduzione al solare fotovoltaico; il secondo capitolo introduce, con l'ausilio 
di alcune fonti reperibili in letteratura, la problematica del monitoraggio degli 
impianti fotovoltaici, soffermandosi sui concetti chiave.
%

%
Il terzo e il quarto capitolo descrivono la soluzione implementata; in particolare, 
il terzo capitolo descrive i meccanismi utilizzati per l'acquisizione dei dati di campo,
il quarto capitolo si sofferma, invece, su come i dati di campo vengono processati 
al fine di produrre \emph{informazioni rilevanti}.
%

%
Il quinto capitolo riporta alcuni esempi di rappresentazione delle \emph{informazioni rilevanti}
che il sistema di monitoraggio implementato \`e in grado di produrre.
%

%
Il sesto, e ultimo, capitolo, infine, suggerisce alcuni possibili temi su cui lavorare
nell'ottica della futura evoluzione del sistema.
%
