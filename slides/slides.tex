\documentclass{beamer}

%\usepackage{graphicx}
%\usepackage{tikz}
%\usepackage[absolute,overlay]{textpos}
%\usepackage{listings}
%\usepackage{textpos}
\usepackage{graphicx}


%\includeonlyframes{1,2,3}


\usetheme{Antibes}
% Orange
\usecolortheme[RGB={32,115,53}]{structure}

\usebackgroundtemplate{
    % UniCT Logo
    \includegraphics[width=370pt]{img/unict.jpg}
}

\newtheorem{samplecode}{Sample Code}

%\setbeamercovered{transparent}
\setbeamertemplate{footline}[frame number]{}
\setbeamertemplate{blocks}[rounded][bg=red]

\newcommand{\tildett}{\raise.17ex\hbox{$\scriptstyle\mathtt{\sim}$}}
% Spaces
\newcommand{\N}{\vskip 0.3 cm}
\newcommand{\n}{\vskip 0.2 cm}
\newcommand{\TAB}{\hskip 1.8 cm}
\newcommand{\tab}{\hskip 0.6 cm}

% Colors
\newcommand{\red}[1]{\textcolor[rgb]{.8,0,0}{#1}}
\newcommand{\blue}[1]{\textcolor[rgb]{0,0,.7}{#1}}
\newcommand{\navy}[1]{\textcolor[rgb]{0,0,.5}{#1}}
\newcommand{\purple}[1]{\textcolor[rgb]{.7,0,.8}{#1}}
\newcommand{\green}[1]{\textcolor[rgb]{0,.6,.1}{#1}}


\title[Un sistema per il monitoraggio di impianti fotovoltaici]{
  Un sistema per il monitoraggio di \\ impianti fotovoltaici
 }\subtitle[]{Progetto e implementazione}
\author{Loris Fichera \n
Relatore: Prof. Corrado Santoro}
\institute[Universit\`a di Catania]{
	Universit\`a degli Studi di Catania\\
        Corso di Laurea Specialistica in Ingegneria Informatica\\
}
\date{20 Luglio 2011}


\begin{document}

% Title Page
\begin{frame}[plain]
  \titlepage
\end{frame}
%

%% \begin{frame}
%% \tableofcontents
%% \end{frame}

%------------------------------
%% \begin{frame}[shrinks,label=o]{Outline}
%% %  \tableofcontents[pausesections]
%% %     \tableofcontents
%%     \begin{columns}
%%      \column{2.0in}
%% %       \tableofcontents[sections={1}]
%% %      \vspace{10mm}
%%        \tableofcontents[sections={2}]
%%       \vspace{10mm}
%%        \tableofcontents[sections={3}]
%%        \column{2.0in}
%%        \tableofcontents[sections={4}] 
%%        \vspace{10mm}
%%       \tableofcontents[sections={5}] 
%% %      \vspace{10mm}
%% %       \tableofcontents[sections={5}]
%%       \end{columns}
%% \end{frame}
%------------------------------

\section{Introduzione}
\begin{frame}{Monitoraggi \emph{immaturi}}
  Il numero di impianti fotovoltaici \emph{grid-connected}, in Italia, \`e in costante aumento
  \begin{itemize}
    \item potenza installata \red{raddoppia} ogni anno
    \item oltre \red{4 GW} al 31/12/2010
  \end{itemize}
  \N
  Il problema del monitoraggio assume importanza per 
  \begin{itemize}
    \item i \green{soggetti responsabili}
    \item gli \green{installatori/manutentori}
  \end{itemize}
  \N
  "Gran parte delle soluzioni oggi in commercio mostrano caratteri di \emph{immaturit\`a}" (C. Podewils 2010)
\end{frame}
%

%
\section{Il monitoraggio di impianti fotovoltaici}
\subsection{Definizione del Problema}
\begin{frame}{Obiettivi}
Un sistema di monitoraggio effettua la \green{raccolta} e \green{l'integrazione} dei dati \red{rilevanti} di un impianto al fine di
determinarne:
%
\begin{itemize}
\item lo stato operativo
\item l'efficienza globale
\item la produzione energetica
\end{itemize}
%
\begin{figure}[!h]
  \begin{center}
    \fbox{\includegraphics[width=170pt]{img/solar_photovoltaic.jpg}}
  \end{center}
\end{figure}
%
\end{frame}
%












%%\section*{Introduction}

%\subsection{}
%
%\begin{frame}{Problem Statement}  
%  \begin{itemize}
%    \item Our work aims at \ldots
%    \item More\dots
%  \end{itemize}
%  \vfill
%  A whitespace gap\\
%  \begin{tiny}
%    Smaller Font
%  \end{tiny}
%\end{frame}
%%


%----------------------------- Core concepts of PROFETA
%% \include{profeta}
%% %----------------------------- PLY features
%% \include{ply}
%% %----------------------------- Propylene description
%% \include{propylene}
%----------------------------- Case study (not useful)
%\include{casestudy}


%% \section{Conclusions}
%% %
%% \begin{frame}{Conclusions}
%% % 3 final remarks
%%   \begin{itemize}
%%     \item \navy{Propylene} is a Python classes generator that extracts
%%     \emph{Profeta Attitudes} from \emph{Profeta Plans}
%% \N
%%     \item \navy{Propylene} uses a pure-Python implementation of \emph{lex}
%%     and \emph{yacc}, \red{PLY}, as parsing tool
%% \N
%%     \item \navy{Propylene} provides a \textbf{graphical representation} 
%%     of the parse tree that is used to generate the code
%%   \end{itemize}
%% \end{frame}
%

%
%% \begin{frame}{References}
%%   \begin{thebibliography}{10}

%%     \beamertemplatearticlebibitems

%% %    \bibitem{bdi}
%% %      Rao, A., Georgeff, M.
%% %      \newblock BDI agents: From theory to practice 
%% %      \newblock {\em Proceedings of the first international 
%% %      conference on multi-agent systems}, 312--319, 1995.

%%     \bibitem{AgentSpeak}
%%       A.S.~Rao
%%       \newblock AgentSpeak(L): BDI agents speak out in a logical 
%%       computable language
%%       \newblock {\em Lecture Notes in Artificial Intelligence}, 
%%       1038:42--55, 1996.
%% \n    
%%     \beamertemplatearrowbibitems
%%     \bibitem{ply}
%%       PLY
%%       \newblock \url{http://www.dabeaz.com/ply/}
%%       \newblock {\em Python Lex-Yacc}
%% \n 
%%     \bibitem{nx}
%%       NetworkX
%%       \newblock \url{http://networkx.lanl.gov/}
%%       \newblock {\em Visual Generator}
 
%%     \end{thebibliography}




%% \end{frame}
%




%
%
\end{document}
