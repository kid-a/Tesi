\documentclass{beamer}

%\usepackage{graphicx}
%\usepackage{tikz}
%\usepackage[absolute,overlay]{textpos}
%\usepackage{listings}
%\usepackage{textpos}
\usepackage{graphicx}
\usepackage{verbatim}


%\includeonlyframes{1,2,3}


\usetheme{Antibes}
% Orange
\usecolortheme[RGB={32,115,53}]{structure}

\usebackgroundtemplate{
    % UniCT Logo
    \includegraphics[width=370pt]{img/unict.jpg}
}

\newtheorem{samplecode}{Sample Code}

%\setbeamercovered{transparent}
\setbeamertemplate{footline}[frame number]{}
\setbeamertemplate{blocks}[rounded][bg=red]

\newcommand{\tildett}{\raise.17ex\hbox{$\scriptstyle\mathtt{\sim}$}}
% Spaces
\newcommand{\N}{\vskip 0.3 cm}
\newcommand{\n}{\vskip 0.2 cm}
\newcommand{\TAB}{\hskip 1.8 cm}
\newcommand{\tab}{\hskip 0.6 cm}

% Colors
\newcommand{\red}[1]{\textcolor[rgb]{.8,0,0}{#1}}
\newcommand{\blue}[1]{\textcolor[rgb]{0,0,.7}{#1}}
\newcommand{\navy}[1]{\textcolor[rgb]{0,0,.5}{#1}}
\newcommand{\purple}[1]{\textcolor[rgb]{.7,0,.8}{#1}}
\newcommand{\green}[1]{\textcolor[rgb]{0,.6,.1}{#1}}


\title[Un sistema per il monitoraggio di impianti fotovoltaici]{
  Un sistema per il monitoraggio di \\ impianti fotovoltaici
 }\subtitle[]{Progetto e implementazione}
\author{Loris Fichera \n
Relatore: Prof. Corrado Santoro}
\institute[Universit\`a di Catania]{
	Universit\`a degli Studi di Catania\\
        Corso di Laurea Specialistica in Ingegneria Informatica\\
}
\date{20 Luglio 2011}


\begin{document}

% Title Page
\begin{frame}[plain]
  \titlepage
\end{frame}
%

%% \begin{frame}
%% \tableofcontents
%% \end{frame}

%------------------------------
%% \begin{frame}[shrinks,label=o]{Outline}
%% %  \tableofcontents[pausesections]
%% %     \tableofcontents
%%     \begin{columns}
%%      \column{2.0in}
%% %       \tableofcontents[sections={1}]
%% %      \vspace{10mm}
%%        \tableofcontents[sections={2}]
%%       \vspace{10mm}
%%        \tableofcontents[sections={3}]
%%        \column{2.0in}
%%        \tableofcontents[sections={4}] 
%%        \vspace{10mm}
%%       \tableofcontents[sections={5}] 
%% %      \vspace{10mm}
%% %       \tableofcontents[sections={5}]
%%       \end{columns}
%% \end{frame}
%------------------------------

\section{Introduzione}
\begin{frame}{Monitoraggi \emph{immaturi}}
  Il numero di impianti fotovoltaici \emph{grid-connected}, in Italia, \`e in costante aumento
  \begin{itemize}
    \item potenza installata \red{raddoppia} ogni anno
    \item oltre \red{4 GW} al 31/12/2010
  \end{itemize}
  \N
  Monitorare gli impianti fotovoltaici diventa importante per
    \begin{itemize}
    \item i \green{soggetti responsabili}
    \item gli \green{installatori/manutentori}
  \end{itemize}
  \N  
  "Gran parte delle soluzioni oggi in commercio mostrano caratteri di \emph{immaturit\`a}" (C. Podewils 2010)
\end{frame}
%

%
\section{Il monitoraggio di impianti fotovoltaici}
\subsection{Definizione del Problema}
\begin{frame}{Obiettivi}
Un sistema di monitoraggio effettua la \green{raccolta} e \green{l'integrazione} dei  \red{dati rilevanti} di un impianto al fine di
determinarne:
%
\begin{itemize}
\item lo stato operativo
\item l'efficienza globale
\item la produzione energetica
\end{itemize}
%
\begin{figure}[!h]
  \begin{center}
    \fbox{\includegraphics[width=140pt]{img/solar_photovoltaic.jpg}}
  \end{center}
\end{figure}
%
Quali sono i \red{dati rilevanti}?
\end{frame}
%

%
\begin{frame}{Classi di utenti (Kolodenny 2008)}
Kolodenny \emph{et al.}, identificano le seguenti \green{classi di utenti}:
\N
%
\begin{itemize}
\item \red{ricercatore}
  \begin{itemize}
  \item quanti pi\`u dati possibile
  \item alto livello di dettaglio
  \end{itemize}
  %
\item \red{proprietario}
  \begin{itemize}
  \item energia prodotta
  \item ritorno economico
  \end{itemize}
  %
\item \red{manutentore}
  \begin{itemize}
  \item comportamento di ogni componente (modulo, inverter...)
  \item dati ambientali
  \end{itemize}
\end{itemize}
%
\end{frame}
%

%
\section{La soluzione implementata}
\begin{frame}{Panoramica del sistema}
%
\begin{figure}[!h]
  \begin{center}
    \fbox{\includegraphics[width=310pt]{img/architecture.png}}
  \end{center}
\end{figure}
%
\begin{itemize}
\item infrastruttura \red{di campo}
\item trasmissione dati via \red{gprs} (o \red{sms})
\item \red{datacenter}
\item accesso ai dati via \red{web services}
\item portale web
\end{itemize}
%
\end{frame}
%

%
\subsection{L'infrastruttura di campo}
\begin{frame}{Infrastruttura di campo}
%
\begin{itemize}
  \item \red{Wireless Sensor Network} (WSN) ZigBee-based
    \begin{itemize}
    \item bassi costi
    \item no cablaggi aggiuntivi
    \item rete autoconfigurante
    \end{itemize}
    \N
  \item previsti tre tipi di nodi
    \begin{itemize}
    \item \red{gateway}
    \item \red{power/inverter transponder}
    \item \red{string transponder}
    \end{itemize}
\end{itemize}
%
\end{frame}
%

%
\begin{frame}{Il Gateway}
\begin{figure}[!h]
  \begin{center}
    \fbox{\includegraphics[width=60pt]{img/gw.jpg}}
  \end{center}
\end{figure}
%
\begin{itemize}
  \item agisce da \red{concentratore dati}
  \item \red{modem gsm} per la trasmissione dei dati
    \begin{itemize}
    \item via \green{sms}
    \item via \green{FTP} \emph{over} \green{gprs}
    \item \red{transmission interval} configurabile
      \end{itemize}
  \item \red{elemento fotovoltaico} per la ricarica della batteria
\end{itemize}
%
\end{frame}
%

%
\begin{frame}{Il Power/Inverter Transponder}
\begin{figure}[!h]
  \begin{center}
    \fbox{\includegraphics[width=60pt]{img/power-transponder.jpg}}
  \end{center}
\end{figure}
%
\begin{itemize}
\item \red{analizzatore di rete} + \red{transponder ZigBee}
\item misura delle grandezze elettriche \\ (corrente, potenza, power factor...)
  \begin{itemize}
  \item a monte del contatore bidirezionale
  \item a valle di ogni inverter
  \item \red{sampling interval} configurabile
  \end{itemize}
\end{itemize}
%
\end{frame}
%

%
\begin{frame}{Lo String Transponder}
\begin{figure}[!h]
  \begin{center}
    \fbox{\includegraphics[width=60pt]{img/string-transponder.jpg}}
  \end{center}
\end{figure}
%
\begin{itemize}
\item \red{sensore di corrente} (a \emph{effetto Hall}) + \red{transponder ZigBee}
\item misura della corrente di stringa
  \begin{itemize}
  \item \red{sampling interval} configurabile  
  \end{itemize}
  \end{itemize}
\end{frame}
%

%
\subsection{Il Datacenter}
%
\begin{frame}{Il Datacenter}
Si occupa di
\begin{itemize}
\item \green{gestire i flussi informativi} generati dai gateway
\item \green{elaborare} e \green{memorizzare} i dati di campo 
\item generare \green{report} e \green{allarmi}
\item fornire \green{accesso ai dati}
\end{itemize}
%
\N
\begin{itemize}
  \item \`e interamente basato su \red{Erlang/OTP}
  \item \`e costituito da un insieme di \red{Erlang applications}
\end{itemize}
%
\begin{figure}[!h]
  \begin{center}
    \fbox{\includegraphics[width=60pt]{img/erlang-logo.png}}
  \end{center}
\end{figure}
%
\end{frame}
%

%
\begin{frame}[fragile]{L'applicazione \texttt{database}}
  Possiede un unico processo \green{worker}, il quale 
  funge da wrapper per il database \red{MySQL} in cui vengono memorizzati
  \begin{itemize}
  \item configurazioni degli impianti
  \item dati di monitoraggio
  \end{itemize}
  %
\begin{exampleblock}{Code snippet}
\begin{semiverbatim}
data_storage:\red{add_new_plant_by_structure} (
      _UniqueID = "MyPlant", 
      _Location = "Nowhere",
      ...
      %% just one inverter
      _Inverters = [{{"YURAKU", "I-3000"}, 
                    "Inverter-00", "0000-0000-0000", 
                    [%% strings here...
                     ]} ]).
\end{semiverbatim}
\end{exampleblock}  
\end{frame}
%

%
%% \begin{frame}{L'applicazione \texttt{sysconf}}
%% \end{frame}
%

%
%% \begin{frame}{L'applicazione \texttt{datamanager}}
%% \end{frame}
%

%
%% \section{Testing}
%% \begin{frame}{Test effettuati}
%% %% il sistema e` stato installato in configurazione minima...
%% \end{frame}
%% %

%% %
%% \section{Sviluppi futuri}
%% \begin{frame}{Sviluppi futuri}


%% \end{frame}












%%\section*{Introduction}

%\subsection{}
%
%\begin{frame}{Problem Statement}  
%  \begin{itemize}
%    \item Our work aims at \ldots
%    \item More\dots
%  \end{itemize}
%  \vfill
%  A whitespace gap\\
%  \begin{tiny}
%    Smaller Font
%  \end{tiny}
%\end{frame}
%%


%----------------------------- Core concepts of PROFETA
%% \include{profeta}
%% %----------------------------- PLY features
%% \include{ply}
%% %----------------------------- Propylene description
%% \include{propylene}
%----------------------------- Case study (not useful)
%\include{casestudy}


%% \section{Conclusions}
%% %
%% \begin{frame}{Conclusions}
%% % 3 final remarks
%%   \begin{itemize}
%%     \item \navy{Propylene} is a Python classes generator that extracts
%%     \emph{Profeta Attitudes} from \emph{Profeta Plans}
%% \N
%%     \item \navy{Propylene} uses a pure-Python implementation of \emph{lex}
%%     and \emph{yacc}, \red{PLY}, as parsing tool
%% \N
%%     \item \navy{Propylene} provides a \textbf{graphical representation} 
%%     of the parse tree that is used to generate the code
%%   \end{itemize}
%% \end{frame}
%

%
%% \begin{frame}{References}
%%   \begin{thebibliography}{10}

%%     \beamertemplatearticlebibitems

%% %    \bibitem{bdi}
%% %      Rao, A., Georgeff, M.
%% %      \newblock BDI agents: From theory to practice 
%% %      \newblock {\em Proceedings of the first international 
%% %      conference on multi-agent systems}, 312--319, 1995.

%%     \bibitem{AgentSpeak}
%%       A.S.~Rao
%%       \newblock AgentSpeak(L): BDI agents speak out in a logical 
%%       computable language
%%       \newblock {\em Lecture Notes in Artificial Intelligence}, 
%%       1038:42--55, 1996.
%% \n    
%%     \beamertemplatearrowbibitems
%%     \bibitem{ply}
%%       PLY
%%       \newblock \url{http://www.dabeaz.com/ply/}
%%       \newblock {\em Python Lex-Yacc}
%% \n 
%%     \bibitem{nx}
%%       NetworkX
%%       \newblock \url{http://networkx.lanl.gov/}
%%       \newblock {\em Visual Generator}
 
%%     \end{thebibliography}




%% \end{frame}
%




%
%
\end{document}
