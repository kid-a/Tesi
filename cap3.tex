\clearpage{\pagestyle{empty}\cleardoublepage}
\chapter{Il datacenter}\label{sec:datacenter}
Il datacenter \`e il componente del sistema delegato alla elaborazione 
del flusso informativo generato dai dispositivi \emph{di campo}.
%
Si tratta di un sistema software costituito da diverse \emph{erlang 
applications}\cite{erlapplication}, ciascuna delle quali implementa 
una o pi\`u delle funzionalit\`a tra quelle riportate di seguito:
%
\begin{itemize}
  \item decodifica dei \emph{dati di campo}
  \item stima di grandezze \emph{aggregate}
  \item memorizzazione dei dati su \emph{memoria persistente}
  \item rilevamento di \emph{condizioni anomale} e generazione di \emph{allarmi}
  \item accesso allo \emph{stato} degli impianti
  \item accesso ai \emph{dati storici} degli impianti
\end{itemize}
%
%% quindi il datacenter e` un componente cruciale: deve mantenere i dati 
%% costantemente aggiornati, deve fornire dei servizi mediante i quali accedere ai 
%% dati, deve essere 'live' h24, 

%% proprieta` desiderabili: liveness, transactions?, robustness
%% scelta della piattaforma: erlang + otp

%% proprieta` desiderabili

%% scelta della tecnologia
\section{Breve introduzione a Erlang/OTP}



%% il datacenter: responsabilita` -> decodifica lettura
%%                                -> gestione database
%%                                -> produzione dati aggregati
%%                                -> produzione allarmi
%%                                -> produzione report sullo stato del sistema
%%                                
%% brevissima introduzione a erlang che spieghi il perche` della scelta
%% architettura: insieme di erlang applications:
\section{L'applicazione \emph{sysconf}}
\section{L'applicazione \emph{datamanager}}
\subsection{Il \emph{gen\_server filter}}
\subsection{Il \emph{gen\_server file\_poller}}
\subsection{La decodifica: \emph{ftp\_protocol}}
%% il supervision tree
\section{L'applicazione \emph{database}}
\section{L'applicazione \emph{datamanager-ws}}
